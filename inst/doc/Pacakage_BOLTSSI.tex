\documentclass[]{article}
\usepackage{lmodern}
\usepackage{amssymb,amsmath}
\usepackage{ifxetex,ifluatex}
\usepackage{fixltx2e} % provides \textsubscript
\ifnum 0\ifxetex 1\fi\ifluatex 1\fi=0 % if pdftex
  \usepackage[T1]{fontenc}
  \usepackage[utf8]{inputenc}
\else % if luatex or xelatex
  \ifxetex
    \usepackage{mathspec}
  \else
    \usepackage{fontspec}
  \fi
  \defaultfontfeatures{Ligatures=TeX,Scale=MatchLowercase}
\fi
% use upquote if available, for straight quotes in verbatim environments
\IfFileExists{upquote.sty}{\usepackage{upquote}}{}
% use microtype if available
\IfFileExists{microtype.sty}{%
\usepackage{microtype}
\UseMicrotypeSet[protrusion]{basicmath} % disable protrusion for tt fonts
}{}
\usepackage[margin=1in]{geometry}
\usepackage{hyperref}
\hypersetup{unicode=true,
            pdftitle={BOLTSSIRR Package for BOLT-SSI: Fully Screening Interaction Effects for Ultra-High Dimensional Data},
            pdfborder={0 0 0},
            breaklinks=true}
\urlstyle{same}  % don't use monospace font for urls
\usepackage{color}
\usepackage{fancyvrb}
\newcommand{\VerbBar}{|}
\newcommand{\VERB}{\Verb[commandchars=\\\{\}]}
\DefineVerbatimEnvironment{Highlighting}{Verbatim}{commandchars=\\\{\}}
% Add ',fontsize=\small' for more characters per line
\usepackage{framed}
\definecolor{shadecolor}{RGB}{248,248,248}
\newenvironment{Shaded}{\begin{snugshade}}{\end{snugshade}}
\newcommand{\KeywordTok}[1]{\textcolor[rgb]{0.13,0.29,0.53}{\textbf{#1}}}
\newcommand{\DataTypeTok}[1]{\textcolor[rgb]{0.13,0.29,0.53}{#1}}
\newcommand{\DecValTok}[1]{\textcolor[rgb]{0.00,0.00,0.81}{#1}}
\newcommand{\BaseNTok}[1]{\textcolor[rgb]{0.00,0.00,0.81}{#1}}
\newcommand{\FloatTok}[1]{\textcolor[rgb]{0.00,0.00,0.81}{#1}}
\newcommand{\ConstantTok}[1]{\textcolor[rgb]{0.00,0.00,0.00}{#1}}
\newcommand{\CharTok}[1]{\textcolor[rgb]{0.31,0.60,0.02}{#1}}
\newcommand{\SpecialCharTok}[1]{\textcolor[rgb]{0.00,0.00,0.00}{#1}}
\newcommand{\StringTok}[1]{\textcolor[rgb]{0.31,0.60,0.02}{#1}}
\newcommand{\VerbatimStringTok}[1]{\textcolor[rgb]{0.31,0.60,0.02}{#1}}
\newcommand{\SpecialStringTok}[1]{\textcolor[rgb]{0.31,0.60,0.02}{#1}}
\newcommand{\ImportTok}[1]{#1}
\newcommand{\CommentTok}[1]{\textcolor[rgb]{0.56,0.35,0.01}{\textit{#1}}}
\newcommand{\DocumentationTok}[1]{\textcolor[rgb]{0.56,0.35,0.01}{\textbf{\textit{#1}}}}
\newcommand{\AnnotationTok}[1]{\textcolor[rgb]{0.56,0.35,0.01}{\textbf{\textit{#1}}}}
\newcommand{\CommentVarTok}[1]{\textcolor[rgb]{0.56,0.35,0.01}{\textbf{\textit{#1}}}}
\newcommand{\OtherTok}[1]{\textcolor[rgb]{0.56,0.35,0.01}{#1}}
\newcommand{\FunctionTok}[1]{\textcolor[rgb]{0.00,0.00,0.00}{#1}}
\newcommand{\VariableTok}[1]{\textcolor[rgb]{0.00,0.00,0.00}{#1}}
\newcommand{\ControlFlowTok}[1]{\textcolor[rgb]{0.13,0.29,0.53}{\textbf{#1}}}
\newcommand{\OperatorTok}[1]{\textcolor[rgb]{0.81,0.36,0.00}{\textbf{#1}}}
\newcommand{\BuiltInTok}[1]{#1}
\newcommand{\ExtensionTok}[1]{#1}
\newcommand{\PreprocessorTok}[1]{\textcolor[rgb]{0.56,0.35,0.01}{\textit{#1}}}
\newcommand{\AttributeTok}[1]{\textcolor[rgb]{0.77,0.63,0.00}{#1}}
\newcommand{\RegionMarkerTok}[1]{#1}
\newcommand{\InformationTok}[1]{\textcolor[rgb]{0.56,0.35,0.01}{\textbf{\textit{#1}}}}
\newcommand{\WarningTok}[1]{\textcolor[rgb]{0.56,0.35,0.01}{\textbf{\textit{#1}}}}
\newcommand{\AlertTok}[1]{\textcolor[rgb]{0.94,0.16,0.16}{#1}}
\newcommand{\ErrorTok}[1]{\textcolor[rgb]{0.64,0.00,0.00}{\textbf{#1}}}
\newcommand{\NormalTok}[1]{#1}
\usepackage{longtable,booktabs}
\usepackage{graphicx,grffile}
\makeatletter
\def\maxwidth{\ifdim\Gin@nat@width>\linewidth\linewidth\else\Gin@nat@width\fi}
\def\maxheight{\ifdim\Gin@nat@height>\textheight\textheight\else\Gin@nat@height\fi}
\makeatother
% Scale images if necessary, so that they will not overflow the page
% margins by default, and it is still possible to overwrite the defaults
% using explicit options in \includegraphics[width, height, ...]{}
\setkeys{Gin}{width=\maxwidth,height=\maxheight,keepaspectratio}
\IfFileExists{parskip.sty}{%
\usepackage{parskip}
}{% else
\setlength{\parindent}{0pt}
\setlength{\parskip}{6pt plus 2pt minus 1pt}
}
\setlength{\emergencystretch}{3em}  % prevent overfull lines
\providecommand{\tightlist}{%
  \setlength{\itemsep}{0pt}\setlength{\parskip}{0pt}}
\setcounter{secnumdepth}{0}
% Redefines (sub)paragraphs to behave more like sections
\ifx\paragraph\undefined\else
\let\oldparagraph\paragraph
\renewcommand{\paragraph}[1]{\oldparagraph{#1}\mbox{}}
\fi
\ifx\subparagraph\undefined\else
\let\oldsubparagraph\subparagraph
\renewcommand{\subparagraph}[1]{\oldsubparagraph{#1}\mbox{}}
\fi

%%% Use protect on footnotes to avoid problems with footnotes in titles
\let\rmarkdownfootnote\footnote%
\def\footnote{\protect\rmarkdownfootnote}

%%% Change title format to be more compact
\usepackage{titling}

% Create subtitle command for use in maketitle
\newcommand{\subtitle}[1]{
  \posttitle{
    \begin{center}\large#1\end{center}
    }
}

\setlength{\droptitle}{-2em}

  \title{`BOLTSSIRR' Package for BOLT-SSI: Fully Screening Interaction Effects
for Ultra-High Dimensional Data}
    \pretitle{\vspace{\droptitle}\centering\huge}
  \posttitle{\par}
    \author{}
    \preauthor{}\postauthor{}
    \date{}
    \predate{}\postdate{}
  

\begin{document}
\maketitle

\section{1 Overview}\label{overview}

This vignette provides an introduction to the `BOLTSSIRR' package.
BOLTSSI is a statistical approach for detecting interaction effects
among predict variables to response variables, which is often an crucial
step in regression modeling of real data for various applications.
Through this publicly available package, we provide a unified
environment to carry out interaction pursuit using a simple sure
screening procedure (SSI) to fully detect significant pure interactions
between predict variables and the response variable in the high or
ultra-high dimensional generalized linear regression models.
Furthermore, we suggest to discretize continuous predict variables, and
utilize the Boolean operation for the marginal likelihood estimates. The
so-called `BOLTSSI' procedure is proposed to accelerate the sure
screening speed of the procedure.

After screening the interaction effects, we just use penalzied
likelihood function with LASSO penalty to further select the variables
including the interaction terms. The objective function is
\[-loglik/nobs+\lambda*penalty.\]

This vignette is organized as follows. Section 2 introduces the basic
principle of our methods. Section 3 illustrates how to screen the
inteaction effecs and choose the variables by using this package.

\section{2 Introduction}\label{introduction}

\subsection{2.1 SSI}\label{ssi}

Assume that given the predictor vector \(x\), the conditional
distribution of the random variable \(Y\) belongs to an exponential
family, whose probability density function has the canonical form
\[f_{Y|x}(y|x)=\exp\{y\theta(x)-b(\theta(x))+c(y)\}\] where \(b(\cdot)\)
and \(c(\cdot)\) are some known functions and \(\theta(x)\) is a
canonical natural parameter. Here we ignore the dispersion parameter
\(\phi\) in the above density function, since we only concentrate on the
estimation of mean regression function. It is well known that the
distributions in the exponential family include the Binomial, Gaussian,
Gamma, Inverse-Gaussian and Poisson distributions.

We consider the following generalized linear model with two-way
interaction:
\[E(Y|X=x)=b'(\theta(x))=g^{-1}\left(\beta_0+\sum_{i=1}^p\beta_iX_i+\sum_{i<j}\beta_{ij}X_iX_j\right)
\] for some link function \(g(\cdot)\). And we focus on the canonical
link function, hence \(g^{-1}(\cdot)=b'\) and
\[\theta(x)=\beta_0+\sum_{i=1}^p\beta_iX_i+\sum_{i<j}\beta_{ij}X_iX_j\triangleq\beta_0+\sum_{i=1}^p\beta_iX_i+\sum_{i<j}\beta_{ij}X_{ij}.\]

Our method is based on the marginal likelihood ratio screening, which
builds on the difference between two marginal negative log-likelihood
functions in the following two models:
\[E(Y|X_i,X_j)=\beta_{i,j0}+\beta_{i,}X_i+\beta_{j,}X_j\] and
\[E(Y|X_i,X_j)=\beta_{i,j0}+\beta_{i,}X_i+\beta_{j,}X_j+\beta_{ij}X_{ij}.\]
To ease the presentation, through this vignette, \(X_{ij}\) is referred
to as an important interaction if its regression coefficient
\(\beta_{ij}\) is nonzero. \(X_i\) is called an active interaction
variable if there exists some \(1 \leq i\neq j \leq p\) such that
\(X_{ij}\) is an important interaction. Our method ``SSI'' is based on
the above difference and to implemnet the screening procedure for
interaction terms.

\subsection{2.2 BOLTSSI}\label{boltssi}

Furthermore, the ``BOLTSSI'' procedure is proposed to accelerate the
sure screening speed of the procedure. The first step of BOLT-SSI is to
discretize one continuous attribute by creating one categorical variabe
wtih a specified number of levels. After discretization, the Boolean
operation can be used to speedup SSI procedure, especially the algorithm
to calculate the difference in the above section 2.1. All details can be
seen in the paper ( et al. {[}2018{]}).

\section{3 Screening and Prediction}\label{screening-and-prediction}

The package can be loaded with the command:

\begin{Shaded}
\begin{Highlighting}[]
\KeywordTok{library}\NormalTok{(BOLTSSIRR)}
\end{Highlighting}
\end{Shaded}

\subsection{\texorpdfstring{3.1 The function
`BOLT\_SSI'}{3.1 The function BOLT\_SSI}}\label{the-function-bolt_ssi}

\subsubsection{Description}\label{description}

This function implements the Sure Independence Screening SSI and BOLTSSI
for interaction screening.

\subsubsection{Usage}\label{usage}

\begin{Shaded}
\begin{Highlighting}[]
\KeywordTok{BOLT_SSI}\NormalTok{(X,y,extra_pairs,code_num,thread_num)}
\end{Highlighting}
\end{Shaded}

\subsubsection{Arguments}\label{arguments}

\begin{longtable}[]{@{}ll@{}}
\toprule
\begin{minipage}[b]{0.12\columnwidth}\raggedright\strut
Arguments\strut
\end{minipage} & \begin{minipage}[b]{0.82\columnwidth}\raggedright\strut
\strut
\end{minipage}\tabularnewline
\midrule
\endhead
\begin{minipage}[t]{0.12\columnwidth}\raggedright\strut
X\strut
\end{minipage} & \begin{minipage}[t]{0.82\columnwidth}\raggedright\strut
The design matrix, of dimensions \(n*p\), without an intercept. Each row
is an observation vector.\strut
\end{minipage}\tabularnewline
\begin{minipage}[t]{0.12\columnwidth}\raggedright\strut
y\strut
\end{minipage} & \begin{minipage}[t]{0.82\columnwidth}\raggedright\strut
The response vector of dimension \(n*1\).\strut
\end{minipage}\tabularnewline
\begin{minipage}[t]{0.12\columnwidth}\raggedright\strut
extra\_pairs\strut
\end{minipage} & \begin{minipage}[t]{0.82\columnwidth}\raggedright\strut
The number of remaining interaction effects by using SSI or
BOLTSSI,default is \(n\).\strut
\end{minipage}\tabularnewline
\begin{minipage}[t]{0.12\columnwidth}\raggedright\strut
code\_num\strut
\end{minipage} & \begin{minipage}[t]{0.82\columnwidth}\raggedright\strut
The level of preditor varibales after discretization, default is
3.\strut
\end{minipage}\tabularnewline
\begin{minipage}[t]{0.12\columnwidth}\raggedright\strut
thread\_num\strut
\end{minipage} & \begin{minipage}[t]{0.82\columnwidth}\raggedright\strut
The number of thread\_num, default is 4.\strut
\end{minipage}\tabularnewline
\bottomrule
\end{longtable}

\subsubsection{Value}\label{value}

Returns a matrix with three columns.

\begin{longtable}[]{@{}ll@{}}
\toprule
\begin{minipage}[b]{0.19\columnwidth}\raggedright\strut
Values\strut
\end{minipage} & \begin{minipage}[b]{0.75\columnwidth}\raggedright\strut
\strut
\end{minipage}\tabularnewline
\midrule
\endhead
\begin{minipage}[t]{0.19\columnwidth}\raggedright\strut
Th frist column\strut
\end{minipage} & \begin{minipage}[t]{0.75\columnwidth}\raggedright\strut
The value \(i\), where \(i\) is the index of the avtive interaction
variable \(X_i\).\strut
\end{minipage}\tabularnewline
\begin{minipage}[t]{0.19\columnwidth}\raggedright\strut
The second column\strut
\end{minipage} & \begin{minipage}[t]{0.75\columnwidth}\raggedright\strut
The value \(j\), where \(j\) is the index of the avtive interaction
variable \(X_j\).\strut
\end{minipage}\tabularnewline
\begin{minipage}[t]{0.19\columnwidth}\raggedright\strut
The third column\strut
\end{minipage} & \begin{minipage}[t]{0.75\columnwidth}\raggedright\strut
The difference between two marginal negative log-likelihood
functions.\strut
\end{minipage}\tabularnewline
\bottomrule
\end{longtable}

\subsubsection{Examples}\label{examples}

\begin{Shaded}
\begin{Highlighting}[]
\KeywordTok{library}\NormalTok{(BOLTSSIRR)}
\KeywordTok{set.seed}\NormalTok{(}\DecValTok{0}\NormalTok{)}
\NormalTok{p=}\DecValTok{300}\NormalTok{;n=}\DecValTok{100}\NormalTok{;rho=}\FloatTok{0.5}
\NormalTok{H<-}\KeywordTok{abs}\NormalTok{(}\KeywordTok{outer}\NormalTok{(}\DecValTok{1}\OperatorTok{:}\NormalTok{p,}\DecValTok{1}\OperatorTok{:}\NormalTok{p,}\StringTok{"-"}\NormalTok{))}
\NormalTok{covxx=rho}\OperatorTok{^}\NormalTok{H}
\NormalTok{cholcov =}\StringTok{ }\KeywordTok{chol}\NormalTok{(covxx)}
\NormalTok{x0 =}\StringTok{ }\KeywordTok{matrix}\NormalTok{(}\KeywordTok{rnorm}\NormalTok{(n}\OperatorTok{*}\NormalTok{p), n, p)}
\NormalTok{x =}\StringTok{ }\NormalTok{x0}\OperatorTok\NormalTok{cholcov}

\CommentTok{#gaussian response}
\KeywordTok{set.seed}\NormalTok{(}\DecValTok{0}\NormalTok{)}
\NormalTok{y=}\DecValTok{2}\OperatorTok{*}\NormalTok{x[,}\DecValTok{1}\NormalTok{]}\OperatorTok{+}\DecValTok{2}\OperatorTok{*}\NormalTok{x[,}\DecValTok{8}\NormalTok{]}\OperatorTok{+}\DecValTok{3}\OperatorTok{*}\NormalTok{x[,}\DecValTok{1}\NormalTok{]}\OperatorTok{*}\NormalTok{x[,}\DecValTok{8}\NormalTok{]}\OperatorTok{+}\KeywordTok{rnorm}\NormalTok{(n)}
\NormalTok{model1=}\KeywordTok{BOLT_SSI}\NormalTok{(x,y)}
\KeywordTok{head}\NormalTok{(model1)}

\CommentTok{#binary response}
\KeywordTok{set.seed}\NormalTok{(}\DecValTok{40}\NormalTok{)}
\NormalTok{feta =}\StringTok{ }\DecValTok{2}\OperatorTok{*}\NormalTok{x[,}\DecValTok{1}\NormalTok{]}\OperatorTok{+}\DecValTok{2}\OperatorTok{*}\NormalTok{x[,}\DecValTok{8}\NormalTok{]}\OperatorTok{+}\DecValTok{3}\OperatorTok{*}\NormalTok{x[,}\DecValTok{1}\NormalTok{]}\OperatorTok{*}\NormalTok{x[,}\DecValTok{8}\NormalTok{]; }
\NormalTok{fprob =}\StringTok{ }\KeywordTok{exp}\NormalTok{(feta)}\OperatorTok{/}\NormalTok{(}\DecValTok{1}\OperatorTok{+}\KeywordTok{exp}\NormalTok{(feta))}
\NormalTok{y =}\StringTok{ }\KeywordTok{rbinom}\NormalTok{(n, }\DecValTok{1}\NormalTok{, fprob)}
\NormalTok{model2=}\KeywordTok{BOLT_SSI}\NormalTok{(x,y)}
\KeywordTok{head}\NormalTok{(model2)}
\end{Highlighting}
\end{Shaded}

\subsection{\texorpdfstring{3.2 The function
`CV\_BOLT\_SSI\_RR'}{3.2 The function CV\_BOLT\_SSI\_RR}}\label{the-function-cv_bolt_ssi_rr}

\subsubsection{Description}\label{description-1}

This function fits a generalized linear model via penalized maximum
likelihood. The regularization path is computed for the lasso penalty at
a grid of values for the regularization parameter lambda. And it
implements Cross-Validation for BOLTSSI. It can deal with all shapes of
data, including very large sparse data matrices.

\subsubsection{Usage}\label{usage-1}

\begin{Shaded}
\begin{Highlighting}[]
\KeywordTok{CV_BOLT_SSI_RR}\NormalTok{(X,y,extra_pairs,cod_num,nfold,nLambda,thread_num)}
\end{Highlighting}
\end{Shaded}

\subsubsection{Arguments}\label{arguments-1}

\begin{longtable}[]{@{}ll@{}}
\toprule
\begin{minipage}[b]{0.12\columnwidth}\raggedright\strut
Arguments\strut
\end{minipage} & \begin{minipage}[b]{0.82\columnwidth}\raggedright\strut
\strut
\end{minipage}\tabularnewline
\midrule
\endhead
\begin{minipage}[t]{0.12\columnwidth}\raggedright\strut
X\strut
\end{minipage} & \begin{minipage}[t]{0.82\columnwidth}\raggedright\strut
The design matrix, of dimensions \(n*p\), without an intercept. Each row
is an observation vector.\strut
\end{minipage}\tabularnewline
\begin{minipage}[t]{0.12\columnwidth}\raggedright\strut
y\strut
\end{minipage} & \begin{minipage}[t]{0.82\columnwidth}\raggedright\strut
The response vector of dimension \(n*1\).\strut
\end{minipage}\tabularnewline
\begin{minipage}[t]{0.12\columnwidth}\raggedright\strut
extra\_pairs\strut
\end{minipage} & \begin{minipage}[t]{0.82\columnwidth}\raggedright\strut
The number of remaining interaction effects by using SSI or
BOLTSSI,default is \(n\).\strut
\end{minipage}\tabularnewline
\begin{minipage}[t]{0.12\columnwidth}\raggedright\strut
code\_num\strut
\end{minipage} & \begin{minipage}[t]{0.82\columnwidth}\raggedright\strut
The level of preditor varibales after discretization, default is
3.\strut
\end{minipage}\tabularnewline
\begin{minipage}[t]{0.12\columnwidth}\raggedright\strut
nfold\strut
\end{minipage} & \begin{minipage}[t]{0.82\columnwidth}\raggedright\strut
The number of folds, default is 10.\strut
\end{minipage}\tabularnewline
\begin{minipage}[t]{0.12\columnwidth}\raggedright\strut
nLambda\strut
\end{minipage} & \begin{minipage}[t]{0.82\columnwidth}\raggedright\strut
The number of lambda, default is 100.\strut
\end{minipage}\tabularnewline
\begin{minipage}[t]{0.12\columnwidth}\raggedright\strut
thread\_num\strut
\end{minipage} & \begin{minipage}[t]{0.82\columnwidth}\raggedright\strut
The number of thread\_num, default is 4.\strut
\end{minipage}\tabularnewline
\bottomrule
\end{longtable}

\subsubsection{Value}\label{value-1}

An object of class ``CV\_BOLTSSI\_RR'' is returned, which is a list with
the ingredients of the crossvalidation fit.

\begin{longtable}[]{@{}ll@{}}
\toprule
\begin{minipage}[b]{0.19\columnwidth}\raggedright\strut
Values\strut
\end{minipage} & \begin{minipage}[b]{0.75\columnwidth}\raggedright\strut
\strut
\end{minipage}\tabularnewline
\midrule
\endhead
\begin{minipage}[t]{0.19\columnwidth}\raggedright\strut
lambdas\strut
\end{minipage} & \begin{minipage}[t]{0.75\columnwidth}\raggedright\strut
The values of lambda used in the fits.\strut
\end{minipage}\tabularnewline
\begin{minipage}[t]{0.19\columnwidth}\raggedright\strut
beta\strut
\end{minipage} & \begin{minipage}[t]{0.75\columnwidth}\raggedright\strut
A nvars x length(lambda) matrix of coefficients, stored in sparse column
format (``CsparseMatrix'').\strut
\end{minipage}\tabularnewline
\begin{minipage}[t]{0.19\columnwidth}\raggedright\strut
lambda\_min\strut
\end{minipage} & \begin{minipage}[t]{0.75\columnwidth}\raggedright\strut
The value of lambda when predicition error arrives at its minimum by
Cross-Validation\strut
\end{minipage}\tabularnewline
\begin{minipage}[t]{0.19\columnwidth}\raggedright\strut
index\_min\strut
\end{minipage} & \begin{minipage}[t]{0.75\columnwidth}\raggedright\strut
The index of optimal value\strut
\end{minipage}\tabularnewline
\begin{minipage}[t]{0.19\columnwidth}\raggedright\strut
covs\strut
\end{minipage} & \begin{minipage}[t]{0.75\columnwidth}\raggedright\strut
The intercept vector of dimension \(n*1\).\strut
\end{minipage}\tabularnewline
\begin{minipage}[t]{0.19\columnwidth}\raggedright\strut
pairs\strut
\end{minipage} & \begin{minipage}[t]{0.75\columnwidth}\raggedright\strut
A vector of all indexes of interaction effects when lambda is condisered
as it optimal value.\strut
\end{minipage}\tabularnewline
\bottomrule
\end{longtable}

\subsubsection{Examples}\label{examples-1}

\begin{Shaded}
\begin{Highlighting}[]
\KeywordTok{set.seed}\NormalTok{(}\DecValTok{0}\NormalTok{)}
\NormalTok{p=}\DecValTok{300}\NormalTok{;n=}\DecValTok{100}
\NormalTok{rho=}\FloatTok{0.5}
\NormalTok{H<-}\KeywordTok{abs}\NormalTok{(}\KeywordTok{outer}\NormalTok{(}\DecValTok{1}\OperatorTok{:}\NormalTok{p,}\DecValTok{1}\OperatorTok{:}\NormalTok{p,}\StringTok{"-"}\NormalTok{))}
\NormalTok{covxx=rho}\OperatorTok{^}\NormalTok{H}
\NormalTok{cholcov =}\StringTok{ }\KeywordTok{chol}\NormalTok{(covxx)}

\NormalTok{x0 =}\StringTok{ }\KeywordTok{matrix}\NormalTok{(}\KeywordTok{rnorm}\NormalTok{(n}\OperatorTok{*}\NormalTok{p), n, p)}
\NormalTok{x =}\StringTok{ }\NormalTok{x0}\OperatorTok\NormalTok{cholcov}
\CommentTok{#gaussian response}
\KeywordTok{set.seed}\NormalTok{(}\DecValTok{0}\NormalTok{)}
\NormalTok{y=}\DecValTok{2}\OperatorTok{*}\NormalTok{x[,}\DecValTok{1}\NormalTok{]}\OperatorTok{+}\DecValTok{2}\OperatorTok{*}\NormalTok{x[,}\DecValTok{8}\NormalTok{]}\OperatorTok{+}\DecValTok{3}\OperatorTok{*}\NormalTok{x[,}\DecValTok{1}\NormalTok{]}\OperatorTok{*}\NormalTok{x[,}\DecValTok{8}\NormalTok{]}\OperatorTok{+}\KeywordTok{rnorm}\NormalTok{(n)}
\NormalTok{model3=}\KeywordTok{CV_BOLT_SSI_RR}\NormalTok{(x,y,}\DataTypeTok{extra_pairs=}\NormalTok{p,}\DataTypeTok{nfold=}\DecValTok{5}\NormalTok{)}


\NormalTok{Lambdas=model3}\OperatorTok{$}\NormalTok{lambdas}
\NormalTok{Beta=model3}\OperatorTok{$}\NormalTok{beta}
\NormalTok{Lambda.min=model3}\OperatorTok{$}\NormalTok{lambda_min}
\NormalTok{index.min=}\KeywordTok{which}\NormalTok{(Lambdas}\OperatorTok{==}\NormalTok{Lambda.min)}
\NormalTok{Pairs=}\KeywordTok{t}\NormalTok{(}\KeywordTok{matrix}\NormalTok{(model3}\OperatorTok{$}\NormalTok{pairs,}\DataTypeTok{nrow =} \DecValTok{2}\NormalTok{))}

\NormalTok{m=}\KeywordTok{length}\NormalTok{(final_Beta)}
\NormalTok{main_index=}\KeywordTok{which}\NormalTok{(final_Beta[}\DecValTok{1}\OperatorTok{:}\NormalTok{p]}\OperatorTok{!=}\DecValTok{0}\NormalTok{)}
\NormalTok{inter_index=}\KeywordTok{which}\NormalTok{(final_Beta[(p}\OperatorTok{+}\DecValTok{1}\NormalTok{)}\OperatorTok{:}\NormalTok{m]}\OperatorTok{!=}\DecValTok{0}\NormalTok{)}
\NormalTok{final_inter_index=Pairs[inter_index,]}
\NormalTok{main_index}
\NormalTok{final_inter_index}
\end{Highlighting}
\end{Shaded}

\subsection{\texorpdfstring{3.3 The function
`BOLT\_Predict'}{3.3 The function BOLT\_Predict}}\label{the-function-bolt_predict}

\subsubsection{Description}\label{description-2}

This function makes predictions from a cross-validated BOLTSSIRR model,
using the stored object, and the optimal value chosed for lambda.

\subsubsection{Usage}\label{usage-2}

\begin{Shaded}
\begin{Highlighting}[]
\KeywordTok{BOLT_Predict}\NormalTok{(x,fit)}
\end{Highlighting}
\end{Shaded}

\subsubsection{Arguments}\label{arguments-2}

\begin{longtable}[]{@{}ll@{}}
\toprule
\begin{minipage}[b]{0.12\columnwidth}\raggedright\strut
Arguments\strut
\end{minipage} & \begin{minipage}[b]{0.82\columnwidth}\raggedright\strut
\strut
\end{minipage}\tabularnewline
\midrule
\endhead
\begin{minipage}[t]{0.12\columnwidth}\raggedright\strut
x\strut
\end{minipage} & \begin{minipage}[t]{0.82\columnwidth}\raggedright\strut
Matrix of new values for \(x\) at which predcitons ate to be made. Must
be a matrix.\strut
\end{minipage}\tabularnewline
\begin{minipage}[t]{0.12\columnwidth}\raggedright\strut
fit\strut
\end{minipage} & \begin{minipage}[t]{0.82\columnwidth}\raggedright\strut
Fitted ``CV\_BOLT\_SSI\_RR'' object.\strut
\end{minipage}\tabularnewline
\bottomrule
\end{longtable}

\subsubsection{Value}\label{value-2}

Returns a predicted response vector .

\subsubsection{Examples}\label{examples-2}

\begin{Shaded}
\begin{Highlighting}[]
\NormalTok{newx=}\KeywordTok{matrix}\NormalTok{(}\KeywordTok{rnorm}\NormalTok{(}\DecValTok{30}\OperatorTok{*}\NormalTok{p),}\DecValTok{30}\NormalTok{)}
\NormalTok{yhat=}\KeywordTok{BOLT_Predict}\NormalTok{(newx,model3)}
\end{Highlighting}
\end{Shaded}


\end{document}
